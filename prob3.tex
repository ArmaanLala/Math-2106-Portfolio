Proof by Contradiction - Homework 12 Question 4


\textbf{Lemma 3.1} If $x \in G$, $x \not \in H$ and $H\leq G$ then $xh \not \in H$ for all $h \in H$

For some group $G$ and subgroup $H$, if $x \in G$ and $x \not \in H$ then $xh \not \in H$ for all $h \in H$.
If we know that $h\in H$, then by the definition of a group, then $h^{-1} \in H$ as well. Assume that $x \not \in H$ and $xh \in H$. Since $H$ is a group, it must be closed under some operation. If we take $xh$ and multiply by $h^{-1}$, then the result should be in $H$ as well. Therefore we get that $xhh^{-1} \in H$. We see that $xhh^{-1} = xe = x$, so we get that $x\in H$ which is a contradiction as we stated that $x \not \in H$ so we can conclude that $xh \not \in H$ for all $h \in H$.


\textbf{Theorem}
Let $A\leq S_4$ be defined by \[A=\{(1),(12)(34),(13)(24),(14)(23),(123),(132),(124),(142),(134),(143),(234),(243)\}.\]  Prove that $A$ has no subgroup of order $6$.   



\begin{proof}
Let $H$ be a subgroup of $A_4$ such that $|H| = 6$. $A_4$ contains the identity element, 3 cycles of order 2, and 8 cycles of order 3. By the pigeonhole principle, at least 2 cycles of order 3 cannot be within $H$. Let us denote these two cycles as $x$ and $y$. By Lagrange's theorem, we define the index of a subgroup to be the number of left/right cosets and can calculate the index by diving the order of the groups. We can then derive that the index of $H$ is $[A_4 : H] = 2$, so we know that there are 2 distinct cosets. By definition, we know that $H$ contains the identity element so $eh \in H$ for all $h\in H$ and $eH = H$. Additionally, if we take $x$ where $x\in A_4$ and $x \not\in H$ then from Lemma 3.1 we can see that $xh \not \in H $  for all $h\in H$ and $xH \neq H$. From this we see that of the 2 cosets, 1 coset is $H$ and the other coset is not equal to $H$. Now let us proceed and look at the cosets generated by $x$ and $y$.

When looking at possible cycles for $x$ and $y$, we again arrive at two distinct possibilities, either that $(x^2 = y \text{ and } x = y^2)$ or that $(x^2 \neq y \text{ and } x \neq y^2)$. Let us look at the first possibility of where $(x^2 = y \text{ and } x = y^2)$. Since $x,y \not\in H$, then $xH \neq H$ and $yH \neq H$. Since we stated that there is only 1 coset not equal to $H$, it follows that $xH = yH = x^2H$. However, if we multiply both sides by $x$, we get $x^2H = xyH = x^3H$. By definition, $x$ is a 3-cycle meaning that $x^3 = e$ so we can reduce our statement to be that $x^2H = eH = H$ so $x^2H = yH = H$ which is a contradiction as we said that $x$ and $y$ are not in $H$.

Let us now look at the other case where $x^2 \neq y \text{ and } x \neq y^2$. Similarly, we see that $xH \neq H$ and $yH \neq H$. Since we know that $x \not\in H$  and $xH \neq H$, $x^2H$ must be equal to either $H$ or $xH$. If $x^2H = H$, then since $x$ is a 3-cycle $x^2 = x^{-1}$ and $x^{-1}\in H$ but in order to fufil the properties of a group, if $x^{-1}\in H$ then $x\in H$ as well which is a contradiction since we stated that $x\not\in H$. If $x^2H \neq H$ then $x^2H = xH$, and by multiplying both sides by $x$, we get that $x^3H = x^2H \rightarrow eH = x^2H \rightarrow H = x^2H$ which is a contradiction since we stated that $x^2H \neq H$.

Therefore, we can see that it is impossible for $H$ to be a subgroup of $A_4$ meaning that it is impossible for $A_4$ to have a subgroup of order 6. 
\end{proof}