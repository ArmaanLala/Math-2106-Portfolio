A proof involving injectivity and surjectivity of a function - Midterm Question 4

\textbf{Theorem}
Let $f: A\rightarrow B$ be a function. For any $Y \subseteq B$, define
$$ f^{-1}(Y): = \{a \in A: f(a) \in Y\} $$
Note that this doesn't require that $f^{-1}$ is a function
Then $f$ is said to be \textbf{midterm injective} if for all $b \in B$
$$|f^{-1} (\{b\})| \leq 1 $$
Prove that f is midterm injective if and only if $f$ is injective
We also have that $f$ is said to be \textbf{midterm surjective} if for all $b \in B$
$$| f^{-1}(\{b\})| > 0 $$
Prove that f is midterm surjective if and only if $f$ is surjective.

Use the above definitions to give a description of a bijection as both midterm injective and midterm surjective.

\begin{proof}
First let us begin by showing that $f$ is midterm injective if and only if $f$ is injective.
Let $f$ be injective. So let $a,b \in A$, then let $f(a),f(b) \in f(A)$, so therefore $f(a),f(b) \in Y$. Then by the definition of injective, each value in A has at most 1 corresponding value in B. Therefore $b\in B$ and $|f^{-1} (b)|$ is either equal to 0 or 1, both of which are less than equal to 1, meaning that $f$ is midterm injective.
Now assume $f$ is midterm injective. We then know that $\forall b \in B  |f^{-1} (b)| \leq 1$. Let $f^{-1} (b) = a$. By the definition of midterm injective, $a$ must be either $a \in A$ or nonexistent which means that it is unique, proving that $f$ is injective. Therefore we have shown that $f$ is midterm injective if and only if $f$ is injective.

Next we will show that $f$ is midterm surjective if and only if $f$ is surjective. Assume $f$ is midterm surjective, then we know that $|f^{-1} (b)|  > 0$, meaning that for all $b \in B$, there exists some $f(a) = b, a \in A$. Sincethere exists some $f(a) = b$ for all $b$, then by definition $f$ is surjective.
Now assume $f$ is surjective, then we know $\forall b \in B, \exists a , f(a) = b$. Since there is always some $a$ that can transform into any $b$, $|f^{-1} (b)|$ is always at least one meaning that $|f^{-1} (b)| > 0$ showing that $f$ is midterm surjective.
Therefore we have shown that $f$ is midterm surjective if and only if $f$ is surjective.

By the definition of bijective, we know that a fucntion must be both injective and surjective to be bijective. Using the definitions of midterm injective and midterm surjective, we can see that for a function to be bijective: 
$$0 < | f^{-1}(\{b\})| \leq 1  $$
$$ | f^{-1}(\{b\})| = 1  $$

This means that for every $b\in B$, there is a unique $a\in A$ and the unique value $a$ is guaranteed to exist.
\end{proof}